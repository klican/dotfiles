\documentclass[paper=a4,fontsize=12pt]{scrartcl}
\usepackage[croatian]{babel}
\usepackage[utf8]{inputenc}
\usepackage[T1]{fontenc}
\usepackage{lmodern}
\usepackage{amsmath,amssymb}
\usepackage{fancyhdr}
\usepackage[font=small,format=plain,labelfont=bf,up]{caption}
\usepackage{graphicx}
\usepackage{rotating}
\usepackage{latexsym}
\usepackage{booktabs}
\usepackage[version=3]{mhchem}
\usepackage{chemmacros}
\renewcommand{\headrulewidth}{0pt}
\pagestyle{fancy}
\rfoot{\small{Sádzané systémom \LaTeXe}}
\setkomafont{sectioning}{\rmfamily\bfseries}

\begin{document}

\noindent
\begin{minipage}{0.3\textwidth}
    Analytická chémia \\
    Cvičenie~č.~<++> \\
    Dátum: \\
    <+dátum+>
\end{minipage}
\begin{minipage}{0.38\textwidth}
    \begin{center}
	\Large{Analýza katiónov}
    \end{center}
\end{minipage}
\begin{minipage}{0.30\textwidth}
    \begin{flushright}
	Andrej Klič \\
	Rozvrhový čas: \\
	14:50--18:20 \\
	stôl~č.~9 
    \end{flushright}
\end{minipage}

\subsection*{Zadanie}
Skupinovými a selektívnymi reakciami dokážte v troch pridelených
vzorkách prítomnosť katiónov 
\ce{Fe^2+ \mbox{, } Fe^3+ \mbox{, } Ni^2+ \mbox{, } Al^3+ \mbox{, } Zn^2+ }.

\subsubsection*{Vzorka~č.~1}
Roztok zelenej farby s neutrálnym \pH{}, bez usadenín a zápachu. 

\begin{flushleft}
    \begin{tabular}[h]{c|c|c|c|c|c|c|c}
 \ce{HCl} & \ce{H2SO4} & \ce{H2S \mbox{/} H+} & \ce{CH3-CS-NH2} & \ce{NaOH} & \ce{NH3} & \ce{KI} & Odhad \\
	\hline
 <++>     & <++>       & <++>                 & <++>            & <++>      & <++>     & <++>    & <++>  \\
	\addlinespace
	\multicolumn{8}{l}{\rule{0pt}{14pt}\small{\textbf{Tabuľka 1:} Skupinové reakcie
    prvej vzorky}}
\end{tabular}
\end{flushleft}

Selektívny dôkaz \ce{Ni^{2+}} :
\begin{reactions}
%    Ni2+ \aq + 2 NH3 \aq + 2 C4H8N2O2 \aq{} &-> Ni(C4H7N2O2)2 \solid + 2 NH4+ \aq \\
\end{reactions}

\subsubsection*{Vzorka~č.~2}
Roztok zelenej farby s neutrálnym \pH{}, bez usadenín a zápachu. 

\begin{flushleft}
    \begin{tabular}[h]{c|c|c|c|c|c|c|c}
 \ce{HCl} & \ce{H2SO4} & \ce{H2S \mbox{/} H+} & \ce{CH3-CS-NH2} & \ce{NaOH} & \ce{NH3} & \ce{KI} & Odhad \\
	\hline
 <++>     & <++>       & <++>                 & <++>            & <++>      & <++>     & <++>    & <++>  \\
	\addlinespace
	\multicolumn{8}{l}{\rule{0pt}{14pt}\small{\textbf{Tabuľka 2:} Skupinové reakcie
    druhej vzorky}}
\end{tabular}
\end{flushleft}


Selektívny dôkaz \ce{Al^{3+}}:
\begin{reactions}
    Al^{3+} + \text{kys.~1,2-dihydroxyantrachinón-3-sulfónová} &->[\mbox{\small{\pH{}>7}}] \text{{\emph{fialový $\odot$}}} \\
\end{reactions}

\subsubsection*{Vzorka~č.~3}
Roztok zelenej farby s neutrálnym \pH{}, bez usadenín a zápachu.

\begin{flushleft}
    \begin{tabular}[h]{c|c|c|c|c|c|c|c}
 \ce{HCl} & \ce{H2SO4} & \ce{H2S \mbox{/} H+} & \ce{CH3-CS-NH2} & \ce{NaOH} & \ce{NH3} & \ce{KI} & Odhad \\
	\hline
 <++>     & <++>       & <++>                 & <++>            & <++>      & <++>     & <++>    & <++>  \\
	\addlinespace
	\multicolumn{8}{l}{\rule{0pt}{14pt}\small{\textbf{Tabuľka 1:} Skupinové reakcie
    prvej vzorky}}
\end{tabular}
\end{flushleft}

Selektívny dôkaz \ce{Hg^{2+}}:
\begin{reactions}
%    Hg^{2+} \aq + 2 KI \aq{} &-> 2 K+ \aq{} + HgI2 \solid \mbox{\emph{\small{\;červeno-oranžová \ce{v} }}} \\ 
%    HgI2 \solid{} + 2 K+ \aq{} &->[\mbox{\emph{\small{nadbytok skúmadla}}}] 2 K+
%    \aq{} + [HgI4]^{2-} \aq{} \mbox{\emph{\small{\;bezfarebný $\odot$}}}  
\end{reactions}
\begin{reactions}
%    Hg^{2+} \aq + 4 KI \aq + Na2SO3 \aq + 2 CuSO4 \aq + HCl \aq{} &-> Cu2[HgI4] \\
%    2 KI \aq + Hg^{2+} \aq + 2 CuI \solid{} &-> Cu2[HgI4] + 2 K+ \aq
\end{reactions}

\subsection*{Záver}
Na tomto laboratórnom cvičení som si vyskúšal skupinové a selektívne reakcie na
dokazovanie katiónov. V pridelených vzorkách som dokázal tieto ióny: \\
\begin{itemize}
    \item 1.~vzorka: \ce{Ni^{2+}} \\
    \item 2.~vzorka: \ce{Ni^{2+}} spolu s \ce{Al^{3+}} \\
    \item 3.~vzorka: \ce{Ni^{2+}} súčasne s \ce{Hg^{2+}} 
\end{itemize}


\end{document}



